\documentclass[journal,transmag]{IEEEtran}

\usepackage[utf8]{inputenc}
\usepackage{hyperref}

\hyphenation{op-tical net-works semi-conduc-tor}

\begin{document}

\title{Esteganografia\\ Sistemas Embarcados}

\author{\IEEEauthorblockN{Jefferson Nunes de Sousa Xavier}}

\markboth{Sistemas Embarcados,~Vol.~1, No.~2, Dezembro~2015}{}

\maketitle

\section{Objetivos}

\textbf{Objetivo Geral:} Criar um sistema cliente/servidor capaz de analizar arquivos de vídeos e extrair imagens econdidas com estaganografia.
\\

\textbf{Objetivos Específicos:}
\begin{itemize}
	\item Extrair imagem estaganografada;
	\item Extrair hash e comparar com o arquivo recebido;
	\item Enviar imagem extraída para um servidor;
\end{itemize}

\section{Introdução}

A esteganografia é a arte de ocultar informações. Inclui métodos que buscam adicionar ou alterar bits em um dado arquivo buscando a melhor forma de impedir sua detecção, esses bits representam uma infromação escondida\footnote{\url{http://www.ic.unicamp.br/\~rocha/pub/papers/segurancaInternetEsteganografia.pdf}}.

Uma grande aplicação da esteganografia está nas marcas d'água em imagens. Pois dá-se na substituição de bits em uma imagem de tal forma que esse efeito é gerado\footnote{\url{https://www.researchgate.net/profile/Hae_Kim2/publication/220162186_Marcas_d\%27gua_Frgeis_de_Autenticao_para_Imagens_em_Tonalidade_Contnua_e_Esteganografia_para_Imagens_Binrias_e_Meio-Tom/links/00463516819b9aa6ff000000.pdf}}.

Este trabalho consiste na análise de arquivos de vídeo com conteúdo esteganografado buscando extrair esse conteúdo. O arquivo original onde se aplica a esteganografia foi separado em blocos de pixels de tamanho 3x3. Onde cada um dos blocos poderá possuir um ou nenhum pixel que tenha sido esteganografado.

Para a extração da imagem esteganografada baseia-se em uma chave que indica qual pixel do bloco atual foi alterado, ou informa que nenhum pixel no bloco foi alterado quando se tem o número zero na chave. Por fim, ao se extrair toda a imagem ainda se têm 128 bits de informação esteganografada que representam o hash md5 do arquivo inicial.

Além do trabalho relacionado a parte de esteganografia ainda foi feito um sistema cliente/servidor com sockets para envio da imagem extraída para um servidor específico. A seguir será apresentada toda a especificação deste trabalho e como foi feita toda a sua implementação.

\section{Especificação}

\begin{itemize}
	\item \textbf{Extrair imagem esteganografada:} A partir de uma arquivo de chave contendo apenas números de zero a nove deve-se extrair a imagem esteganografada. Estes números represental qual pixel no bloco foi alterado e vem ser pegados uma quantidade específica de bits menos significativos desse pixel. A chave passada é circular, ou seja, ao se chegar ao final volta-se a primeira posição e pode-se continuar o processo de extração. Isso é feito até que sejam extraídos todos os pixels da imagem escondida. Cada sequência de bits que é extraída deve ser completada com zeros bucando ter-se uma quantidade de oito bits para se formar um byte.
	\item \textbf{Extrair hash md5:} Ao se extrair toda a imagem escondida o processo de extração continua para a extração do hash. Nesse processo os bits que são estraídos devem ser concatenados até se alcançar 128 bits. Nessa etapa os bits que são estraídos devem ser substituídos com zeros no arquivo principal.
	\item \textbf{Comparar hashs:} Com as duas etapas anteriores completas têm-se a imagem extraída e o hash do arquivo principal, a partir disso deve-se gerar o hash do arquivo principal e uma comparação desse hash gerado com o hash extraído deve ser feita, garantindo assim a integridade do arquivo recebido.
	\item \textbf{Enviar imagem extraída:} Se a comparação dos hashs retornar um valor verdadeiro deve-se enviar a imagem extraída para um servidor, esse envio é feito com utilização de sockets.
\end{itemize}

\section{Implementação e Prototipação}

Os sistemas foram desenvolvidos usando a linguagem C++ e serão detalhados a seguir.

\begin{itemize}
	\item \textbf{Servidor:} O servidor foi construído apenas para receber as imagens que são extraídas pelo cliente. O mesmo é composto da função principal que é responsável por inicializar os parâmetros de ip e porta em que o servidor estará executando e abrir uma nova conexão. Toda lógica relacionada a comunicação entre os sistemas está desenvolvida na classe de conexão que é responsável por iniciar os parâmetros de conexão no lado servidor, ou seja, inicializar o socket, fazer o bind, o listen e aceitar novas conexões de clientes. A partir disso, sempre que um novo cliente tem sua conexão aceita o servidor entra no método para receber a imagem e faz as leituras dos bytes e os escreve em um arquivo de saída. Ao final das leituras o arquivo de saída estará preenchido totalmente com a imagem que foi extraída e enviada pelo cliente.

	\item \textbf{Cliente:} O lado do cliente é responsável pela parte da esteganografia, extraindo a imagem e o hash e enviado essa imagem extraída para o servidor. A seguir serão detalhados cada um dos pontos principais desenvolvidos para realização desse processo.

	\begin{itemize}
		\item \textbf{Extração da imagem:} Para a realização da extração da imagem esteganografada são necessários algums pontos importantes. Inicialmente os parâmetros como quantidade de bits que foram esteganografados para cada pixel, as dimensões do arquivo principal e as dimensões do arquivo de saída devem ser conhecidas, esses parâmetros são passados no momento de execução do programa. A partir desse ponto a primeira tarefa a ser executada é o carregamento da chave para uma string, a chave é importante pois é usada para saber quais são os pixels que estão esteganografados em cada bloco do arquivo principal, com a chave carregada pode-se então começar a extração da imagem e o programa vai passando pela chave circulamente até que se atinja a leitura completa de todos os bytes da imagem escondida.
		O processo de extração dá-se passando pelo arquivo principal em blocos de tamanho 3x3, lê-se o valor indicado na chave na posição atual que representa qual o pixel do bloco atual foi esteganografado, a partir disso pega-se o byte e os seus N bits menos significativos, o número N é indicado no início da execução do programa como foi dito anteriormente. Os bits são salvos em um vetor e completa-se este vetor com zeros até que se atinja o tamanho de oito bits, após isso pega-se os bits representados no vetor e estes são transformados em um byte que é escrito em um arquivo de saída. Esse mesmo processo se repete até que o arquivo de saída esteja completo com a imagem extraída.
		\item \textbf{Extração e comparação do hash:} O processo continua para a extração do hash, porém, os bits não são mais completados com zeros e sim concatenados, dessa forma, a cada oito bits lidos esses são transformados em bytes. Ao final, um vetor com 16 bytes é gerado contendo os bytes do hash que foi extraído. Além disso, no processo de extração do hash os bits lidos são zerados no arquivo principal. Para completar, é gerado o hash do novo arquivo principal com os bits do hash gerados e esse hash gerado é comparado com o hash que foi extraído, caso a comparação seja verdadeira entra-se no processo de envio da imagem extraída.
		\item \textbf{Envio da imagem:} Para realização do envio da imagem é feito inicialmente uma conexão com o servidor. O método de envio lê o arquivo que foi gerado na extração da imagem de 4096 em 4096 bytes e envia ao servidor o número 4096 (para que este saiba a quantidade de bytes que serão lidos) e os bytes lidos. Esse envio está em loop até que todo o arquivo tenha sido lido.
		\item \textbf{Sinal de interrupção:} Existe também, o tratamento do sinal de interrupção no cliente. Este por sua vez quando é recebido imprime na tela a porcentagem do progresso de extração, completa o arquivo com os bytes 0x00 e envia este arquivo ao servidor.
	\end{itemize}
\end{itemize}

\section{Conclusão}

Este trabalho trouxe a aplicação prática de um dos conceitos estudados na disciplina de Sistemas Embarcados. Esse conceito foi explorado na estração de imagens escondidas em arquivos de vídeos, esta esteganografia foi aplicada usando-se blocos 3x3 de pixels em que um ou nenhum pixel foi esteganografado de acordo com uma chave específica. A partir dessa chave pode-se analisar o arquivo e extrair as imagens esteganografadas.

Além da imagem estraiu-se também um hash de criptografia md5 do arquivo enviado que também estava esteganografado, este hash pode comprovar que o arquivo recebido não foi alterado.

Foram utilizados sockets também no sistema cliente servidor para envio das imagens extraídas.

Todo o código dos sistemas desenvolvidos pode ser encontrado em: \url{https://github.com/jeffersonxavier/steganography}.

\end{document}
