\documentclass[journal,transmag]{IEEEtran}

\usepackage[utf8]{inputenc}
\usepackage{hyperref}

\hyphenation{op-tical net-works semi-conduc-tor}

\begin{document}

\title{Esteganografia\\ Sistemas Embarcados}

\author{\IEEEauthorblockN{Jefferson Nunes de Sousa Xavier}}

\markboth{Sistemas Embarcados,~Vol.~1, No.~2, Dezembro~2015}{}

\maketitle

\section{Objetivos}

\textbf{Objetivo Geral:} Criar um sistema cliente/servidor capaz de analizar arquivos de vídeos e extrair imagens econdidas com estaganografia.
\\

\textbf{Objetivos Específicos:}
\begin{itemize}
	\item Extrair imagem estaganografada;
	\item Extrair hash e comparar com o arquivo recebido;
	\item Enviar imagem extraída para um servidor;
\end{itemize}

\section{Introdução}

\section{Especificação}

\section{Implementação e Prototipação}

\section{Conclusão}

Este trabalho trouxe a aplicação prática de um dos conceitos estudados na disciplina de Sistemas Embarcados. Esse conceito foi explorado na estração de imagens escondidas em arquivos de vídeos, esta esteganografia foi aplicada usando-se blocos 3x3 de pixels em que um ou nenhum pixel foi esteganografado de acordo com uma chave específica. A partir dessa chave pode-se analisar o arquivo e extrair as imagens esteganografadas.

Além da imagem estraiu-se também um hash de criptografia md5 do arquivo enviado que também estava esteganografado, este hash pode comprovar que o arquivo recebido não foi alterado.

Foram utilizados sockets também no sistema cliente servidor para envio das imagens extraídas.

Todo o código dos sistemas desenvolvidos pode ser encontrado em: \url{https://github.com/jeffersonxavier/steganography}.

\end{document}
